\documentclass[a4paper,
fontsize=11pt,
%headings=small,
oneside,
numbers=noperiodatend,
parskip=half-,
bibliography=totoc,
final
]{scrartcl}

\usepackage{synttree}
\usepackage{graphicx}
\setkeys{Gin}{width=.4\textwidth} %default pics size

\graphicspath{{./plots/}}
\usepackage[ngerman]{babel}
\usepackage[T1]{fontenc}
%\usepackage{amsmath}
\usepackage[utf8x]{inputenc}
\usepackage [hyphens]{url}
\usepackage{booktabs} 
\usepackage[left=2.4cm,right=2.4cm,top=2.3cm,bottom=2cm,includeheadfoot]{geometry}
\usepackage{eurosym}
\usepackage{multirow}
\usepackage[ngerman]{varioref}
\setcapindent{1em}
\renewcommand{\labelitemi}{--}
\usepackage{paralist}
\usepackage{pdfpages}
\usepackage{lscape}
\usepackage{float}
\usepackage{acronym}
\usepackage{eurosym}
\usepackage[babel]{csquotes}
\usepackage{longtable,lscape}
\usepackage{mathpazo}
\usepackage[normalem]{ulem} %emphasize weiterhin kursiv
\usepackage[flushmargin,ragged]{footmisc} % left align footnote

\usepackage{listings}

\urlstyle{same}  % don't use monospace font for urls

\usepackage[fleqn]{amsmath}

%adjust fontsize for part

\usepackage{sectsty}
\partfont{\large}

%Das BibTeX-Zeichen mit \BibTeX setzen:
\def\symbol#1{\char #1\relax}
\def\bsl{{\tt\symbol{'134}}}
\def\BibTeX{{\rm B\kern-.05em{\sc i\kern-.025em b}\kern-.08em
    T\kern-.1667em\lower.7ex\hbox{E}\kern-.125emX}}

\usepackage{fancyhdr}
\fancyhf{}
\pagestyle{fancyplain}
\fancyhead[R]{\thepage}

%meta
%meta

\fancyhead[L]{K. Schuldt \\ %author
LIBREAS. Library Ideas, 30 (2016). % journal, issue, volume.
\href{http://nbn-resolving.de/
}{}} % urn
\fancyhead[R]{\thepage} %page number
\fancyfoot[L] {\textit{Creative Commons BY 3.0}} %licence
\fancyfoot[R] {\textit{ISSN: 1860-7950}}

\title{\LARGE{Rezension zu: Smiljana Antonijević (2015). \emph{Amongst Digital Humanists: An Ethnographic Study of Digital Knowledge Production}. Basingstoke ; New York : Palgrave Macmillan}} %title %title
\author{Karsten Schuldt} %author

\setcounter{page}{1}

\usepackage[colorlinks, linkcolor=black,citecolor=black, urlcolor=blue,
breaklinks= true]{hyperref}

\date{}
\begin{document}

\maketitle
\thispagestyle{fancyplain} 

%abstracts

%body
Smiljana Antonijević verspricht im Titel ihres Buches eine
ethnographische Studie zu Forschenden in den Digital Humanities. Als
Forschungsprojekt ist das ein sinnvolles Unterfangen. Die Digital
Humanities werden, je nach Fall als vorhergesagter Trend oder als schon
institutionalisierte Forschungsrichtung, seit einigen Jahren als
Begründung für strategische Planungen von Forschungseinrichtungen und
Forschungsförderinstitutionen verwendet, forschungspolitisch werden sie
als Zielsetzung für die wissenschaftspolitische Steuerung genutzt,
Hochschulbibliotheken scheinen sich zum Teil sehr stark an den
vorgeblichen Anforderungen dieser Forschenden zu orientieren -- und
gleichzeitig ist nicht klar, was genau diese Forschenden tun. Dabei war
Antonjjević für dieses Projekt -- das ihre Promotion darstellt -- gut
positioniert, da sie in den Niederlanden selber in Digital
Humanities-Projekten tätig war. Ihre Forschung betrieb sie dort und in
den USA. Dies schränkt die Aussagekraft ihrer Studie geographisch ein;
da die Digital Humanities allerdings zumeist für Forschungslandschaften
im Globalen Norden konzipiert werden, scheint diese Einschränkung
vertretbar. Die Studie basiert vor allem auf Beobachtungen und
Interviews, die zwischen 2010 und 2013 in 23 Institutionen in Europa und
den USA mit 258 Partizipierenden durchgeführt wurden. Für eine
ethnographische, also stark qualitativ orientierte Forschung, ist dies
beachtlich. Was die Studie einschränkt, ist der Drang der Autorin, die
Digital Humanities nicht nur zu untersuchen, sondern gleichzeitig
verbessern zu wollen. Dies scheint ihr den Blick auf die Widersprüche,
die sie selber aufdeckt, zu verstellen.

Was das Buch nicht ist, ist eine Darstellung des gesamten
ethnographischen Forschungsprozesses dieses Promotionsprojektes. Die
Autorin deutet nur kurz an, wie sie bei den Interviews und Beobachtungen
vorging und zitiert oft direkt aus den Interviews. Grundsätzlich aber
ist das Buch eine schnell zu lesende Zusammenfassung der
Forschungsergebnisse.

\section*{Das Herkommen der Digital
Humanities}\label{das-herkommen-der-digital-humanities}

In einem ersten, kurzen Kapitel stellt die Autorin die Entstehung der
Digital Humanties aus den \enquote{Computational linguistics} dar, die
-- so ihre Erklärung -- ab ungefähr 2005 in die \enquote{Digital
Humanties} transformiert wurden, verbunden mit neuen Versprechen und
Vorstellungen. Diese Darstellung -- die nicht unbedingt von allen in den
Digital Humanities Aktiven geteilt wird (vergleiche Thompson Klein 2015,
Burdick et al. 2012) -- erklärt zum Beispiel den Fokus der meisten
Digital Humanities-Projekte auf Text Corpora und auf Verfahren, die in
der Linguistik genutzt werden können, obgleich Humanities
selbstverständlich weit mehr Felder (und Medientypen, mit oder an denen
geforscht wird) umfasst. Gleichzeitig ist es eine in anderen Punkten
recht unkritische Darstellung, die nicht darauf eingeht, was diese
Projekte in den Universitäten und Forschungseinrichtungen eigentlich
tatsächlich verändern. Diese unkritische Haltung, welche zwar die
Tätigkeiten der untersuchten Forschenden objektiv darstellt, aber
gleichzeitig die Vorstellung von den Digital Humanities als Zukunft der
Humanities nicht diskutiert, findet sich im gesamten Buch. Mehr noch:
Die Autorin geht gerade in diesem, ersten Kapitel auf einige wenige
Kritiken an den Digital Humanities ein -- aber nur auf einige, was den
Eindruck hinterlässt, dass sie auf andere Kritiken nicht antworten will
-- und widerspricht diesen. Insbesondere postuliert sie, dass diese
Projekte nicht im Übermass Forschungsgelder für Infrastrukturprojekte
binden würden, die ansonsten in anderer geisteswissenschaftlicher
Forschung eingesetzt würde, sondern dass Forschungsgelder in den
gesamten Humanities immer zu wenig und deshalb umkämpft sind.

\section*{Die banalen digitalen
Tätigkeiten}\label{die-banalen-digitalen-tuxe4tigkeiten}

In den Interviews orientierte sich die Autorin an einem
Forschungs-Workflow (Collect, Find, Analyze, Write, Communicate,
Organize, Annotate, Cite, Reflect, Archive, Share), den sie zum Beispiel
den interviewten Forschenden als idealtypischen Verlauf einer
Forschungsaktivität vorlegte, um mit ihnen im Interview über diesen zu
diskutieren. Der Workflow, als Kreislauf konzipiert, ähnelt dem im
Bibliothekswesen oft zitieren Research Data Lifecycle, ist aber nicht
mit diesem identisch. Angesichts der grossen Versprechen beziehungsweise
Ankündigungen, dass die Digital Humanities die Forschungsprozesse und
-fragen in den Humanities radikal ändern würden (siehe zum Beispiel
Burdick et al. 2012), ist das, was die Interviews und Beobachtungen
konkret zeigen, erstaunlich banal. Die Forschenden recherchieren digital
und setzen zum Schreiben digitale Werkzeuge ein, wobei immer wieder
Google-Produkte im Vordergrund stehen. Einige Forschende annotieren
PDF-Dokumente. Ansonsten werden digitale Werkzeuge nur in Ausnahmefällen
benutzt. So fand die Autorin zum Beispiel keine verbreitete Praxis der
Analyse von Daten mit digitalen Instrumenten. Auch
Literaturverwaltungssoftware wurde kaum eingesetzt, obwohl die meisten
Forschenden Einführungen für diese besucht hatten. Unterschiede
bestanden zwischen Geisteswissenschaften und Naturwissenschaften
(Sciences), aber nur in Teilbereichen. So schreiben viele Forschende in
den Humanities mit Word, während viele in den Sciences -- die zum Teil
als Kontrollgruppe befragt wurden -- LaTeX verwenden. Aber die grossen,
datengetriebenen Forschungsprojekte, von denen in der Literatur zu den
Digital Humanities seit einigen Jahren als zukünftige Praxis berichtet
wird (und auf die sich beispielsweise Bibliotheken ausrichten), fand die
Autorin quasi nicht.

Der grosse Unterschied zur Forschungspraxis früherer Jahrzehnte scheint
die bessere Recherchierbarkeit und Verfügbarkeit von Dokumenten zu sein,
nicht etwa das Entstehen von neuen Paradigmen oder Forschungsfragen.

Die Frage des Teilens von Daten (Sharing) evoziert interessante
Ergebnisse, da die Haltung dazu vor allem im US-amerikanischen
Universitätssystem von der Position der Forschenden abzuhängen scheint.
Grundsätzlich sind Forschende zum Teilen bereit und haben auch bestimmte
Praxen entwickelt (allerdings oft für sich allein, nicht in Gruppen oder
Institutionen). Dies gilt jedoch nicht, wenn sie auf eine akademische
Karriere hinarbeiten und sich in sogenannten Tenure Tracks (also
befristeten Stellen, bei denen sie regelmässig evaluiert werden und
diese Evaluationsergebnisse über die weitere Beschäftigung entscheiden)
befinden. Auf diesen Stellen, die zum Teil auch in europäischen
Universitäten eingeführt wurden (insbesondere in Form von
Junior-Professuren), um die Qualität der Professorinnen und Professoren
sicherzustellen, verwehren sich die Forschenden dem Teilen von
Forschungsdaten. Offenbar sind sie -- durch das akademische System -- so
sehr auf die eigenen Konkurrenzvorteile bedacht, dass sie im Gegensatz
zu anderen Forschenden -- die zumeist auf festen Stellen angestellt sind
-- ihre Forschungen verschliessen. Die Autorin bespricht dies nicht
weiter, aber es erinnert sehr an die Feststellungen von Richard Münch
(2011), dass der \enquote{akademische Kapitalismus} die Qualität der
Forschung bedroht.

Grundsätzlich stellen die in der Studie befragten Forschenden fest, dass
sie sich die digitalen Fähigkeiten bis hin zum Programmieren selber
beigebracht hätten. Ganz nachvollziehbar ist das nicht, da auch die
Angebote von Bibliotheken und Digital Humanities Center erwähnt werden.
Die Autorin verweigert zu Recht, das Feld zu glätten und stellt eher den
Widerspruch dar. Vorwerfen kann man ihr, dass sie es ohne weitere
Diskussion dieser Differenz tut. Allerdings verweisen die Forschenden
auch darauf, dass sie sich oft von ihren Departements nicht dabei
unterstützt sehen, neue Tools und Fähigkeiten zu erlernen, eine Aussage,
die sich vor allem auf Arbeitszeit, die (nicht) zur Verfügung gestellt
wird, bezieht und nicht etwa auf Weiterbildungsangebote. Ein ähnlicher
Widerspruch ist der von Forschenden geäusserte Wunsch, dass es Tools
geben sollte, die den Forschungsprozess ineinandergreifend digital
gestatten, also zum Beispiel die Recherche in Datenbanken, die Analyse,
das Schreiben und Teilen von Daten, in einem Tool, während sie
gleichzeitig die Tools, die sie kennen, selbst dann, wenn sie
unzufrieden sind, weiter verwenden, solange sie damit ihre Aufgaben
erfüllen können. Gleichzeitig gab es diese Tools als \enquote{Virtuelle
Forschungsumgebungen} für einige Jahre in grosser Zahl, ohne das sie von
vielen Forschenden genutzt wurden. Die reine Aussage, dass sie sich
solche Tools wünschen würden, ist offenbar keine Aussage darüber, was
sie wirklich tun. An diesem Punkt wäre es von Vorteil gewesen, wenn die
Autorin vertiefend die Entscheidungsprozesse der Forschenden für oder
gegen Tools untersucht hätte. (Vergleiche dazu eher Bender 2016)

Einen anderen Unterschied, den die Autorin erkennt, ist der, dass
jüngere Forschende eher dazu tendieren, digitale Tools zu nutzen (aber
es geht dabei weiterhin nicht vorrangig um Tools zur Analyse von Daten,
sondern zum Beispiel um Google-Docs), als ältere; wobei ältere
Forschende trotzdem wissenschaftlich erfolgreich sind.

Die Autorin präsentierte diese Ergebnisse, aber es fehlt zumeist eine
weitere Diskussion. Bei den wenigen weitergehenden Darstellungen scheint
sie dahin zu tendieren, erklären zu wollen, wie sich die Vorstellungen
von den Digital Humanities dennoch umsetzen lassen. Allerdings ist nicht
klar, wieso. Ihre Ergebnisse scheinen eher in eine andere Richtung zu
deuten, nämlich dahin, dass die Digital Humanities in der
Forschungsrealität keine wirklichen Auswirkungen haben, sondern dass
sich eher da, wo es möglich ist, langsam digitale Tools durchsetzen und
dass das Internet -- inklusive der Digitalisierungsprojekte in
Bibliotheken -- eine schnellere, umfassendere Recherche und
Zugänglichkeit erlaubt, als zuvor. Dies ist nicht zu missachten, aber es
widerlegt die Vorstellungen von den angeblich radikal neuen Digital
Humanities aus der Sicht der Personen, die davon am meisten betroffen
sein sollten. Es scheint eher der digitale Wandel zu sein, den die
Autorin beschreibt und der sich auch in der restlichen Gesellschaft
beachten lässt, keinesfalls aber eine neue Wende in den Humanities. Wie
gesagt, kommt die Autorin nicht zu diesem Schluss, sondern interpretiert
eher, dass das Feld der Digital Humanities sich langsam,
forschungsfeld-bezogen unterschiedlich, entwickeln würde. Dem
Rezensenten scheint diese Interpretation nicht nachvollziehbar; aber es
ist der Autorin anzurechnen, dass sie Ergebnisse so darstellt, dass
unterschiedliche Interpretationen möglich sind.

\section*{Bibliotheken und Digital Humanities
Center}\label{bibliotheken-und-digital-humanities-center}

Die Aktivitäten von Bibliotheken im Bezug auf die Digital Humanities
gehen an den Forschenden nicht unbemerkt vorbei. Sie haben aber nicht
den Einfluss, der sich in der bibliothekarischen Literatur zu den
Digital Humanities erhofft wird. (Siehe zum Beispiel Hartsell-Gundy,
Braunstein \& Golomb 2015) Viele Forschende haben Erfahrungen mit
spezifischen Veranstaltungen, die von Bibliotheken für Forschende in den
Digital Humanities angeboten werden. An einer Stelle geht die Autorin
darauf ein, dass eine Anzahl von Forschenden diese sehr positiv
bewerten, aber ein Grossteil ihnen eher eine beschränkte Sinnhaftigkeit
zuschreibt. (Antonijević 2015:76) Bibliotheken würden, so die
Forschenden, in den Veranstaltungen zu sehr auf Fragen des Bestandes
fokussiert sein und dann, wenn es zu Forschungsfragen kommt, keine
weiterführende Informationen anbieten können. Der Austausch mit direkten
Fachkolleginnen und -kollegen sei für die Praxis weit hilfreicher.

Ausführlicher zum Thema werden Bibliotheken im Kapitel zu Digital
Humanities Center behandelt, von denen die Autorin im Rahmen ihrer
Forschung insgesamt elf besucht hat. Viele dieser Center sind einer
Bibliothek angegliedert -- zum Teil auch räumlich -- oder in diese
integriert. Dabei stellt die Autorin fest, dass die ersten dieser Center
schon in den 1980er Jahren gegründet wurden, also keine neue Entwicklung
darstellen, und bis in den 1990er Jahre Bottom-Up-Initiativen
darstellten, während sie seitdem vor allem Top Down von
Universitätsleitungen oder im Rahmen von Forschungsförderungen gegründet
werden. Antonijević beschreibt die Arbeit dieser Center als langsam aber
stetig. Es scheint kaum Einrichtungen zu geben, die einen Ansturm von
Forschenden erleben. Viele Einrichtungen scheinen auch keine klare
Auffassung davon zu haben, was Ihre genaue Aufgabe sein soll. Sie
müssten fast immer auf die Forschenden zu gehen und versuchen, diese
davon zu überzeugen, dass die Center sie bei ihrer Arbeit unterstützen
können. In der Interpretation der Darstellung dieser Arbeit weicht die
Wahrnehmung des Rezensenten wieder massiv von der der Autorin ab. Sie
beschreibt diese Arbeit positiv, macht sich teilweise Gedanken dazu, wie
sie verbessert werden könnte. Für den Rezensenten scheint ersichtlich,
dass die beschriebenen Digital Humanities Center keine wirklichen
Aufgaben für die Forschung erfüllen, sondern vor allem aufgrund
missgeleiteter Annahmen und Wünschen von Verwaltungen und
Fördereinrichtungen bestehen. Es muss aber erneut positiv betont werden,
dass die Autorin ihre Darstellung so gestaltet hat, dass mehrere
Interpretationen möglich sind.

Bemerkenswert ist, dass sie aber selber die Anbindung an die
Bibliotheken als ein mögliches Problem für die Etablierung der Center in
der Forschungspraxis ansieht. Durch diese Zuordnung würden die Center
von den Forschenden als Teil der Infrastruktureinrichtungen und nicht
der Forschungspraxis angesehen werden. (Vergleiche für eine andere
Darstellung und Interpretation der nicht wirklich funktionierenden
Integration solcher Top-Down etablierten Center in den Forschungsbetrieb
Thompson Klein 2015)

\section*{Fazit}\label{fazit}

Der Titel der Studie und auch die Einleitung lassen eine tiefergehende
Darstellung der Praxis der Digital Humanities erwarten, als sie
letztlich geliefert werden. Dennoch ist die Darstellung erfrischend
offen und transparent. Die Autorin deutet ihre Ergebnisse selber sehr in
eine Richtung -- als Möglichkeit der Digital Humanities und gerade der
Digital Humanities Center, besser zu werden --, die sich nicht wirklich
aus den Ergebnissen herzuleiten scheint. Sichtbar wird in den
Beschreibungen eher, dass es die Digital Humanities gar nicht in der
Weise gibt, wie sie gerne beschrieben werden. Neue Fragestellungen,
Forschungspraxen oder Paradigmen werden nicht erkennbar, sondern
vielmehr Forschende, die sich, wenn sie das selber als sinnvoll ansehen,
digitaler Tools bedienen. Daneben existiert ein Diskurs um Digital
Humanities, der nicht wirklich von der Forschung, sondern von
Fördereinrichtungen und Universitätsleitungen geführt wird, sich zwar in
Digital Humanities Center manifestiert, aber kaum Einfluss auf die
tatsächliche Forschungspraxis hat. Die in der Literatur oft postulierte
Verstärkung der interdisziplinären Zusammenarbeit zeigt sich quasi
nicht. Entgegen ihrer eigenen Interpretation scheint Antonijević viele
der Vermutungen, die aus kritischen Perspektiven über die Digital
Humanities angestellt werden, zu bestätigen.

Wie angedeutet, ist das Buch in einem leicht zugänglichen Englisch
geschrieben und lädt vor allem ein, sich selber, über die allfälligen
Beispielsammlungen und Manifeste zur Digital Humanities hinaus, ein Bild
vom (geisteswissenschaftlichen) Forschungsalltag im digitalen Zeitalter
zu machen, der bezogen auf die Nutzung digitaler Tools, recht banal und
-- vorausgesetzt, man wurde zuvor von den Versprechen und Ankündigungen
mitgerissen -- ernüchternd erscheint.

\section*{Literatur}\label{literatur}

Bender, Michael. \emph{Forschungsumgebungen in den Digital Humanities :
Nutzerbedarf, Wissenstransfer, Textualität} (Sprache und Wissen; 22).
Berlin ; Boston: De Gruyter, 2016

Burdick, Anne ; Drucker, Johanna ; Lunenfeld, Peter ; Presner, Todd ;
Schnapp, Jeffrey (2012). \emph{Digital\_Humanities}. Cambridge\,;
London: The MIT Press, 2012

Hartsell-Gundy, Arianne ; Braunstein, Laura ; Golomb, Liorah (edit.)
(2015). \emph{Digital Humanities in the Library: Challenges and
Opportunities for Subject Specialists}. Chicago: American Library
Association, 2015

Münch, Richard (2011). \emph{Akademischer Kapitalismus: Über die
politische Ökonomie der Hochschulreform} (Edition Suhrkamp, 2633).
Berlin: Suhrkamp Verlag, 2011

Thompson Klein, Julie (2015). \emph{Interdisciplining digital
humanities: boundary work in an emerging field} (Digital Humanities).
Ann Arbor: University of Michigan Press, 2015

%autor
\begin{center}\rule{0.5\linewidth}{\linethickness}\end{center}

\textbf{Karsten Schuldt} (Chur / Berlin). Wissenschaftlicher Mitarbeiter
Schweizerisches Institut für Informationswissenschaft, HTW Chur;
Lehrbeauftragter FH Potsdam, Redakteut LIBREAS. Library Ideas.

\end{document}
